\documentclass[10pt,a4paper]{article}
\usepackage[utf8]{inputenc}
\usepackage[french]{babel}
\usepackage[T1]{fontenc}
\usepackage{amsmath}
\usepackage{amsfonts}
\usepackage{amssymb}
\usepackage{graphicx}
\usepackage{tikz}
\usepackage[left=2cm,right=2cm,top=2cm,bottom=2cm]{geometry}
\author{J.Riton}
\title{Correction TD 1 Algèbre}
\begin{document}
\newcounter{mycpt}
\setcounter{mycpt}{1}

\section{Correction}
\paragraph{Exercice \arabic{mycpt}\addtocounter{mycpt}{1}}
\subparagraph{1}

a)

La proposition $\text{non}(P \text{ et } Q)$ et $(non\, P) \text{ ou }(\text{non } Q)$ sont équivalentes si ces deux propositions sont vraies en même temps, c'est-à-dire qu'elles présentent les mêmes tables de vérités.

La table de vérité de $\text{non}(P\text{ et } Q)$ est
$$
\begin{array}{|l|l|c|c|}
P & Q & P \text{ et } Q & \text{non}(P\text{ et }Q)\\
\hline
0 & 0 & 0 & 1\\
0 & 1 & 0 & 1\\
1 & 0 & 0 & 1\\
1 & 1 & 1 & 0\\
\end{array} 
$$

\newcommand{\ou}{\text{ ou }}
\newcommand{\non}{\text{non }}
\newcommand{\et}{\text{ et }}
La table de vérité de $((\non P)\ou (\non Q))$ est
$$
\begin{array}{|l|l|c|c|c|}
P & Q & \non P & \non Q & \non P\ou \non Q\\
\hline
0 & 0 & 1 & 1 & 1\\
0 & 1 & 1 & 0 & 1\\
1 & 0 & 0 & 1 & 1\\
1 & 1 & 0 & 0 & 0\\
\end{array} 
$$

b)

La proposition $(P \ou (Q\et R))$ et $(( P \ou Q) \et(P \ou R)$ sont équivalentes si ces deux propositions sont vraies en même temps, c'est-à-dire qu'elles présentent les mêmes tables de vérités.

La table de vérité de $(P \ou (Q\et R))$ est
$$
\begin{array}{|l|l|c|c|c|}
P & Q & R  & (Q \et R) & (P \ou (Q\et R)) \\
\hline
0&0 & 0 & 0 & 0\\
0&0 & 1 & 0 & 0\\
0&1 & 0 & 0 & 0\\
0&1 & 1 & 1 & 1\\
1&0 & 0 & 0 & 1\\
1&0 & 1 & 0 & 1\\
1&1 & 0 & 0 & 1\\
1&1 & 1 & 1 & 1\\
\end{array} 
$$

La table de vérité de $(( P \ou Q) \et(P \ou R))$ est
$$
\begin{array}{|l|l|c|c|c|c|}
P & Q &  R  & P\ou Q & P\ou R & (( P \ou Q) \et(P \ou R))\\
\hline
0&0 & 0 & 0 & 0 & 0\\
0&0 & 1 & 0 & 1 & 0\\
0&1 & 0 & 1 & 0 & 0\\
0&1 & 1 & 1 & 1 & 1\\
1&0 & 0 & 1 & 1 & 1\\
1&0 & 1 & 1 & 1 & 1\\
1&1 & 0 & 1 & 1 & 1\\
1&1 & 1 & 1 & 1 & 1\\
\end{array} 
$$


\subparagraph{2)}
$ P \Rightarrow Q $ a pour valeur de vérité Faux (ou 1) si et seulement si $P$ et vrai et que $Q$ est fausse.
Ainsi  la négation de $ P \Rightarrow Q $ est $P et non(Q)$.

On peut le vérifier avec les tables de vérité:
$$
\begin{array}{|l|l|c|c|c|}
P & Q  & \non Q &  P\et \non Q\\
\hline
0 & 0 & 1 & 0 \\
0 & 1 & 0 & 0 \\
1 & 0 & 1 & 1 \\
1 & 1 & 0 & 0 \\
\end{array} 
$$
et
$$
\begin{array}{|l|l|c|c|c|}
P & Q  & P\Rightarrow Q &   \non (P\Rightarrow Q)\\
\hline
0 & 0 & 1 & 0 \\
0 & 1 & 1 & 0 \\
1 & 0 & 0 & 1 \\
1 & 1 & 1 & 0 \\
\end{array} 
$$


\paragraph{Exercice \arabic{mycpt}\addtocounter{mycpt}{1}}
\subparagraph{1.}
L'implication a pour valeur de vérité "vrai" lorsque $a=1$ et $b=2$:
$"a<b"$ est vrai et $"a^2<b^2$ est vrai aussi. 

L'implication a pour valeur de vérité "vrai" lorsque $a=4$ et $b=3$:
$"a<b"$ est fausse et $"a^2<b^2$ est fausse aussi.

L'implication a pour valeur de vérité "fausse" lorsque $a=-1$ et $b=0$:
$"a<b"$ est vrai et $"a^2<b^2$ est fausse aussi.

\subparagraph{2.}
$$
\begin{array}{ccc}
\underbrace{P} & \Rightarrow & \underbrace{Q} \\
"hypoth\grave ese" &   &   "conclusion" \\
\end{array}
$$

L'implication précédente est incorrecte (hypothèse vraie et conclusion fausse pour certaines valeurs de $a$ et de $b$).




$a=-1$ et $b=0$ constitue un contre-exemple.

Cette implication devient correcte si l'on restreint les paramètres $a$ et $b$ à des nombres positifs (ou à des sous-ensembles de nombres positifs). 

En effet, 
$$ b^2>a^2 \Leftrightarrow b^2-a^2>0 \Leftrightarrow (b-a)(b+a)>0 \Leftrightarrow (b-a>0 \et a+b>0) \ou (b-a<0 \et b+a<0)$$

La chaîne de proposition ci-dessus est vérifié dès que le dernier maillon est vérifié.
On a un "ou" : disjonction de cas , comme $"b-a<0"$ n'est jamais vérifié il faut que $"b-a>0 \et a+b>0"$ le soit.
On a un "et" : conjonction, comme  $"b-a>0"$ est toujours vérifié (c'est l'hypothèse), il faut et il suffit que $"a+b>0"$ soit vérifiée. (on peut aussi avoir $a\in \left\lbrace 1\right\rbrace $et $b\in \left] -1,+\infty\right[$  ou
une partie dans la surface grisée ci-dessous (dessiner la partie $\left\lbrace a+b>0 \right\rbrace $).


\begin{tikzpicture}

\end{tikzpicture}

\paragraph{Exercice \arabic{mycpt}\addtocounter{mycpt}{1}}
\subparagraph{a)}

Ici, il faut raisonner.

On suppose que $P$ est vraie et il faut démontrer que $Q$ est vrai.
 

\section{Porte logique}
\subsection{Information}
Codage NRZ : on a une tension minimum et une tension maximum: disons 0V et 5V.
La tension au sein d'une cellule d'information varie entre le minimum et le maximum.
Une tension en deça d'un certain seuil (disons 1V) donne l'information 0.
Une tension au dela d'un certain seuil (disons 4V) donne l'information 1.
\subsection{Porte logique}
Une porte logique est un circuit recevant un ou plusieurs bit en entrée et renvoie un bit en sortie.
\subsection{Transistor}
\begin{tikzpicture}
\draw (0,0) -- (3,0) -- (3,1);
\draw (3.5,1.25) -- (3.75,1.25) -- (3.75,2);
\draw (3.5,.75) -- (3.75,.75) -- (3.75,0);
\draw (0,0) -- (3,0) -- (3,-1);
\draw (3.5,-1.25) -- (3.75,-1.25) -- (3.75,-2);
\draw (3.5,-.75) -- (3.75,-.75) -- (3.75,0);
\draw (3.75, 0) -- (5.75,0);
\end{tikzpicture}
\end{document}