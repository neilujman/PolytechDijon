\documentclass[10pt,a4paper]{article}
\usepackage[utf8]{inputenc}
\usepackage[french]{babel}
\usepackage[T1]{fontenc}
\usepackage{amsmath}
\usepackage{amsfonts}
\usepackage{amssymb}
\usepackage{graphicx}
\usepackage{tikz}
\usepackage[left=2cm,right=2cm,top=2cm,bottom=2cm]{geometry}
\author{J.Riton}
\title{Correction TD 1 Algèbre}
\begin{document}
\newcounter{mycpt}
\setcounter{mycpt}{1}

\section{Correction}
\paragraph{Exercice \arabic{mycpt}\addtocounter{mycpt}{1}}
\subparagraph{1}

a)

La proposition $\text{non}(P \text{ et } Q)$ et $(non\, P) \text{ ou }(\text{non } Q)$ sont équivalentes si ces deux propositions sont vraies en même temps, c'est-à-dire qu'elles présentent les mêmes tables de vérités.

La table de vérité de $\text{non}(P\text{ et } Q)$ est
$$
\begin{array}{|l|l|c|c|}
P & Q & P \text{ et } Q & \text{non}(P\text{ et }Q)\\
\hline
0 & 0 & 0 & 1\\
0 & 1 & 0 & 1\\
1 & 0 & 0 & 1\\
1 & 1 & 1 & 0\\
\end{array} 
$$

\newcommand{\ou}{\text{ ou }}
\newcommand{\non}{\text{non }}
\newcommand{\et}{\text{ et }}
La table de vérité de $((\non P)\ou (\non Q))$ est
$$
\begin{array}{|l|l|c|c|c|}
P & Q & \non P & \non Q & \non P\ou \non Q\\
\hline
0 & 0 & 1 & 1 & 1\\
0 & 1 & 1 & 0 & 1\\
1 & 0 & 0 & 1 & 1\\
1 & 1 & 0 & 0 & 0\\
\end{array} 
$$

b)

La proposition $(P \ou (Q\et R))$ et $(( P \ou Q) \et(P \ou R)$ sont équivalentes si ces deux propositions sont vraies en même temps, c'est-à-dire qu'elles présentent les mêmes tables de vérités.

La table de vérité de $(P \ou (Q\et R))$ est
$$
\begin{array}{|l|l|c|c|c|}
P & Q & R  & (Q \et R) & (P \ou (Q\et R)) \\
\hline
0&0 & 0 & 0 & 0\\
0&0 & 1 & 0 & 0\\
0&1 & 0 & 0 & 0\\
0&1 & 1 & 1 & 1\\
1&0 & 0 & 0 & 1\\
1&0 & 1 & 0 & 1\\
1&1 & 0 & 0 & 1\\
1&1 & 1 & 1 & 1\\
\end{array} 
$$

La table de vérité de $(( P \ou Q) \et(P \ou R))$ est
$$
\begin{array}{|l|l|c|c|c|c|}
P & Q &  R  & P\ou Q & P\ou R & (( P \ou Q) \et(P \ou R))\\
\hline
0&0 & 0 & 0 & 0 & 0\\
0&0 & 1 & 0 & 1 & 0\\
0&1 & 0 & 1 & 0 & 0\\
0&1 & 1 & 1 & 1 & 1\\
1&0 & 0 & 1 & 1 & 1\\
1&0 & 1 & 1 & 1 & 1\\
1&1 & 0 & 1 & 1 & 1\\
1&1 & 1 & 1 & 1 & 1\\
\end{array} 
$$


\subparagraph{2)}
$ P \Rightarrow Q $ a pour valeur de vérité Faux (ou 1) si et seulement si $P$ et vrai et que $Q$ est fausse.
Ainsi  la négation de $ P \Rightarrow Q $ est $P et non(Q)$.

On peut le vérifier avec les tables de vérité:
$$
\begin{array}{|l|l|c|c|c|}
P & Q  & \non Q &  P\et \non Q\\
\hline
0 & 0 & 1 & 0 \\
0 & 1 & 0 & 0 \\
1 & 0 & 1 & 1 \\
1 & 1 & 0 & 0 \\
\end{array} 
$$
et
$$
\begin{array}{|l|l|c|c|c|}
P & Q  & P\Rightarrow Q &   \non (P\Rightarrow Q)\\
\hline
0 & 0 & 1 & 0 \\
0 & 1 & 1 & 0 \\
1 & 0 & 0 & 1 \\
1 & 1 & 1 & 0 \\
\end{array} 
$$


\paragraph{Exercice \arabic{mycpt}\addtocounter{mycpt}{1}}
\subparagraph{1.}
L'implication a pour valeur de vérité "vrai" lorsque $a=1$ et $b=2$:
$"a<b"$ est vrai et $"a^2<b^2$ est vrai aussi. 

L'implication a pour valeur de vérité "vrai" lorsque $a=4$ et $b=3$:
$"a<b"$ est fausse et $"a^2<b^2$ est fausse aussi.

L'implication a pour valeur de vérité "fausse" lorsque $a=-1$ et $b=0$:
$"a<b"$ est vrai et $"a^2<b^2$ est fausse aussi.

\subparagraph{2.}
$$
\begin{array}{ccc}
\underbrace{P} & \Rightarrow & \underbrace{Q} \\
"hypoth\grave ese" &   &   "conclusion" \\
\end{array}
$$

L'implication précédente est incorrecte (hypothèse vraie et conclusion fausse pour certaines valeurs de $a$ et de $b$).




$a=-1$ et $b=0$ constitue un contre-exemple.

Cette implication devient correcte si l'on restreint les paramètres $a$ et $b$ à des nombres positifs (ou à des sous-ensembles de nombres positifs). 

En effet, 
$$ b^2>a^2 \Leftrightarrow b^2-a^2>0 \Leftrightarrow (b-a)(b+a)>0 \Leftrightarrow (b-a>0 \et a+b>0) \ou (b-a<0 \et b+a<0)$$

La chaîne de proposition ci-dessus est vérifié dès que le dernier maillon est vérifié.
On a un "ou" : disjonction de cas , comme $"b-a<0"$ n'est jamais vérifié il faut que $"b-a>0 \et a+b>0"$ le soit.
On a un "et" : conjonction, comme  $"b-a>0"$ est toujours vérifié (c'est l'hypothèse), il faut et il suffit que $"a+b>0"$ soit vérifiée. (on peut aussi avoir $a\in \left\lbrace 1\right\rbrace $et $b\in \left] -1,+\infty\right[$  ou
une partie dans la surface grisée ci-dessous (dessiner la partie $\left\lbrace a+b>0 \right\rbrace $).


\begin{tikzpicture}

\end{tikzpicture}

\paragraph{Exercice \arabic{mycpt}\addtocounter{mycpt}{1}}
\subparagraph{a)}

Ici, il faut raisonner.

On suppose que $P$ est vraie et il faut démontrer que $Q$ est vrai.

Supposons alors que la proposition s'écrivant $"\forall x\in \mathbb{R}, f(x)=0"$ soit vraie.
Ainsi, la propriété $f(x)=0$ est valide pour tout $x\in \mathbb{R}$. Ce qui veut dire que l'image par la fonction $f$ de $x$ vaut $0$ pour tout $x$ réel. En particulier, pour $x=0$. Ainsi il existe au moins un $x$ pour lequel son image est $0$.

\subparagraph{b)}
On veut ici la réciproque de $P\Rightarrow Q$. \`A savoir que $Q\Rightarrow P$.
On veut savoir si l'existence d'un réel pour lequel son image par $f$ est $0$ implique que tous les autres nombres réels ont pour image $0$.
Ce qui est le cas lorsque nous considérons la fonction nulle
$$f:\mathbb{R}\rightarrow \mathbb{R}, x\mapsto 0$$.
Si $f$ n'est pas la fonction nulle, alors il existe un nombre pour lequel son image est non nul, c'est-à-dire que $P$ est mis en défaut.

\subparagraph{c)}
Lorsque $Q$ vraie, il y a un certain $x_0\in\mathbb{R}$ tel que $f(x_0)=0$. $f(x_0)$ ne vérifie alors ni $f(x_0)<0$ ni $f(x_0)>0$. Ce qui met en défaut $R$. Ainsi, dans $Q\Rightarrow R$ l'hypothèse $Q$ est vraie sans que la conclusion $R$ le soit. Donc $Q\Rightarrow R$ n'est pas exacte.

\subparagraph{d)}
$"non(R)"$ se traduit par $"\exists x \in \mathbb{R} , f(x) \geq 0 et \exists x\in \mathbb{R} f(x) \leq 0"$.

Soit alors $x_1 \in \mathbb{R}$ tel que $f(x_1)\geq 0$ . Si $f(x_1)=0$ alors $x_1$ convient pour vérifier $Q$. Sinon $f(x_1)>0$. 
Maintenant, soit $x_2\in \mathbb{R}$ tel que $f(x_2)\leq 0$. Si $f(x_2)=0$ alors $x_2$ convient pour vérifier $Q$. Sinon $f(x_2)<0$.

Il nous reste donc le cas où ni $f(x_1)=0$ ni $f(x_2)=0$. On a donc $f(x_1)>0$ et $f(x_2)<0$.

Par hypothèse, $f$ est continue. Le théorème des valeurs intermédiaires affirme donc qu'il existe $x_0$ dans l'intervalle ouvert borné par $x_1$ et $x_2$ tel que $f(x_0)=0$.


\subparagraph{e)}
$"non(Q)"$ exprime $\forall x \in \mathbb{R}, f(x)\neq 0$.

$"non(P)"$ exprime $\exists x \in \mathbb{R}, f(x)\neq 0$.

On raisonne comme pour $"P\Rightarrow Q"$.

\section{Porte logique}

\subsection{Algèbre de Boole}
Approche algébrique de la logique.

$B$ est un ensemble à deux éléments noté $B=\left\lbrace V, F \right\rbrace$.


\subsection{Information}
Codage NRZ : on a une tension minimum et une tension maximum: disons 0V et 5V.
La tension au sein d'une cellule d'information varie entre le minimum et le maximum.
Une tension en deça d'un certain seuil (disons 1V) donne l'information 0.
Une tension au dela d'un certain seuil (disons 4V) donne l'information 1.
\subsection{Porte logique}
Une porte logique est un circuit recevant un ou plusieurs bit en entrée et renvoie un bit en sortie.
\subsection{Transistor}
\begin{tikzpicture}
\draw (0,0) -- (3,0) -- (3,1);
\draw (3.5,1.25) -- (3.75,1.25) -- (3.75,2);
\draw (3.5,.75) -- (3.75,.75) -- (3.75,0);
\draw (0,0) -- (3,0) -- (3,-1);
\draw (3.5,-1.25) -- (3.75,-1.25) -- (3.75,-2);
\draw (3.5,-.75) -- (3.75,-.75) -- (3.75,0);
\draw (3.75, 0) -- (5.75,0);
\end{tikzpicture}
\end{document}