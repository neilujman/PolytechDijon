\documentclass[10pt,a4paper]{article}
\usepackage[utf8]{inputenc}
\usepackage[french]{babel}
\usepackage[T1]{fontenc}
\usepackage{amsmath}
\usepackage{amsfonts}
\usepackage{amssymb}
\usepackage{makeidx}
\usepackage{graphicx}
\usepackage[left=2cm,right=2cm,top=2cm,bottom=2cm]{geometry}
\author{J.Riton}
\title{Analyse S1 - Correction - TD 1}
\date{1er semestre 2024-2025}
\begin{document}
\maketitle

\newcounter{mycpt}
\setcounter{mycpt}{1}

\paragraph{Exercice \arabic{mycpt} \addtocounter{mycpt}{1}}

\subparagraph{1.}

On cherche à isoler $x$.

On utilise les propriétés de la droite réelle ordonnée.

On a 
$$
3x\leq x-5 \Leftrightarrow 2x \leq -5 \Leftrightarrow x\leq -\dfrac{2}{5}
$$

\subparagraph{2.}

On fait un tableau de signes.

\paragraph{Exercice \arabic{mycpt} \addtocounter{mycpt}{1}}

\subparagraph{1.}
On retourne à la définition de la valeur absolue.

On a 
$$
|x|=\left\lbrace\begin{array}{lll}
 & x & \text{ si } x\geq 0,\\
   & -x &  \text{ si } x < 0,
\end{array} \right.
$$
et

$$
|x+3|=\left\lbrace\begin{array}{lll}
 & x+3 & \text{ si } x+3\geq 0,\\
   & -x-3 &  \text{ si } x+3 < 0,
\end{array} \right.
$$

Tableau :

$$
\begin{array}{l|lllllllll}
x &    -\infty  &     & -3       &   & 0        &   & +\infty \\
\hline 
|x| &    \vline & -x  &  \vline  & -x  & \vline & x     & \Vert   \\
\hline
|x+3| &    \vline & -x-3  &  \vline  & x+3  & \vline & x+3     & \vline    
 
 

\end{array}
$$


\end{document}
